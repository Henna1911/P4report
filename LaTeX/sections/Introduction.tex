\chapter{Introduction}

\section{Initial Problem Statement}

\subsection{Motivation}

It is possible to use voice effects while performing. Many effects exist, and it is possible to change the parameters of an effect to one's liking in real time. 
A problem can be changing an effect and/or effect parameters while performing. This could be because the effect pedal is on the floor, or somewhere else out of reach. Another problem could be the lack of knobs to turn, or buttons to press.

\subsection{Statement}

\textit{What existing technologies are there for altering voice effects in real-time, and how do they work?}


\subsection{Research Questions}

\begin{itemize}
	\item What are the most common voice effects?
	\begin{itemize}
		\item What are the limits of the effects
	\end{itemize}
	\item Does any existing technology use body movement or sensors to apply effects?
\end{itemize}

\subsection{Target Group}
The criteria for the target group in this project are:

\begin{itemize}
	\item Solo singers
	\item Choir
	\item Band singers
	\item Has to know effects, and effect parameters
\end{itemize}

The target group consists of singers that know about voice effects. They should not play an instrument while singing because they must be able to use their arms or hands for controlling the effects. There is no specific genre or type of singer as the only criteria is that they know the technicalities behind singing.

