\chapter{Introduction}\label{ch:Intro}

\section{Initial Problem Statement}

It is possible to use voice effects while performing. Many useful effects for performing exist, and it is possible to change the parameters of these effects to one's liking in real time. 
A problem can be applying an effect and/or changing the effect parameters while performing without having a big control board in front of the performer which may cause a disruption of the performance.

\subsection{Statement}

\textit{How does one create a system that applies voice effects to a voice in real-time, while also having an intuitive interface design?}


\subsection{Research Questions}\label{sub:ResearchQ}

\begin{itemize}
	\item What are the most common voice effects?
	\begin{itemize}
		\item Which of these effects would performers have the desire to change during a performance?
	\end{itemize}
	\item Does any existing technology use body gestures or sensors to apply effects?
	\item Which gesture should be used with which effect, and how?
\end{itemize}

\subsection{Target Group}
The criteria for the target group in this project are:

\begin{itemize}
	\item They should have experience with performing 
\end{itemize}

The target group consists of performers that know about voice effects. They should not play an instrument or similar while performing, because they must be able to use their body for controlling the effects. There is no specific genre or type of performer as the only criteria is that they know the technicalities behind performing.

People who fulfil these criteria could be singers, stage actors and stand-up comedians.