\chapter{Introduction}\label{ch:Intro}

\section{Initial Problem Statement}

The problem addressed in this project deals with real-time access to voice effects for performers. 
Some performers finds it useful to use voice effectswhile performing. This project focuses on making the effects accessible and interchangeable in real-time. Instead of most commonly only being applied in set amounts by the push of a button, it could be beneficial for the performer to be able to tamper with the effect parameters in real-time also. In many cases it is not the performer that controls which effects are applied and when, rather it is applied by someone backstage or someone else. \\

Additionally, many existing devices used for adding voice effects can be difficult to use, say during a concert. This could be a pedal, which often just has set amounts of effects, where the performer pushes a button and e.g. a harmony is applied. The problem with the pedal is that it is stationary and the performer will have to reach for it to apply an effect, which is inconvenient in some cases.\\

In this project, it is the desire to give the performer full control of the effects and its parameters, while moving freely across the stage.

It could be interesting to make a wearable device that grants the performer control of the effects, while also being intuitive for the performer to use.


\subsection{Statement}

\textit{How does one create a wearable device that applies voice effects to a voice in real-time, while also having an intuitive interface design?}


\subsection{Research Questions}\label{sub:ResearchQ}

\begin{itemize}
	\item What are the most common voice effects?
	\begin{itemize}
		\item Which of these effects would performers have the desire to change during a performance?
	\end{itemize}
	\item Does any existing technology use body gestures or sensors to apply effects?
	\item Which gesture should be used with which effect, and how?
\end{itemize}

\subsection{Target Group}
The criteria for the target group in this project are:

\begin{itemize}
	\item They should have experience with performing 
\end{itemize}

The target group consists of performers that know about voice effects. They should not play an instrument or similar while performing, because they must be able to use their body for controlling the effects. There is no specific genre or type of performer as the only criteria is that they know the technicalities behind performing.

People who fulfil these criteria could be singers, stage actors and stand-up comedians.