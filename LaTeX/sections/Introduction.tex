\chapter{Introduction}\label{ch:Intro}

To use voice effects when performing has become more and more common through recent years. The voice effects grants the performers a wide range of opportunities. Each effect has its own unique property, and each have both pros and cons. An example of this is auto-tune.
The pros of auto-tune is that a singer can sound like they're singing in tune, pitch perfect and with perfect timing, but the downsides of this could be making everything too perfect\citep{Tyrangiel_2009}.

If the performer wants to use voice effects during their performance they must plan it all very carefully and have pre-sets they can activate or have another person control them during the performance. This removes a lot of control from the performer, which could be a hindrance for some.

This project focuses on making the effects readily available for the performer in real-time. This will be done by making a wearable device, which the performer can use to apply and change voice effects.

Initially, some research will create some understanding of what is needed to create a proper device, that a performer would use during a performance.

A glove will be made for the performer to apply the effects via gestures. The design will be focused around making the device intuitive and easy to use.
When implementing the prototype, the Arduino and Pure Data platforms will be used alongside an Arduino circuit board, to be able to communicate between the audio processing of Pure Data and the physical device via the Arduino.
Two voice effects will be implemented with appropriate gestures and parameters. While only two will be implemented at this stage, the system will be able to support even more effects.

The prototype will be tested for technical faults, to be sure that it is functional, and it will be tested by users to learn whether the device is intuitive to use.

This data will then be analysed both quantitatively and qualitatively to give a broader view of the results. 

Based on the results from the evaluation, a discussion and conclusion will be made.


\section{Initial Problem Statement}

The problem addressed in this project deals with real-time access to voice effects for performers. 
Some performers finds it useful to use voice effects while performing. This project focuses on making the effects accessible and interchangeable in real-time. Instead of most commonly only being applied in set amounts by the push of a button, it could be beneficial for the performer to be able to tamper with the effect parameters in real-time also. In many cases it is not the performer that controls which effects are applied and when, rather it is applied by someone backstage or someone else. \\

Additionally, many existing devices used for adding voice effects can be difficult to use, say during a concert. This could be a pedal, which often just has set amounts of effects, where the performer pushes a button and e.g. a harmony is applied. The problem with the pedal is that it is stationary and the performer will have to reach for it to apply an effect, which is inconvenient in some cases.\\

In this project, it is the desire to give the performer full control of the effects and its parameters, while moving freely across the stage.

It could be interesting to make a wearable device that grants the performer control of the effects, while also being intuitive for the performer to use.


\subsection{Statement}
In order to obtain the necessary information regarding the subject, the following statement has been formed.

\textit{How does one create a wearable device that applies voice effects to a voice in real-time, while also having an intuitive interface design?}


\subsection{Target Group}
The criteria for the target group in this project are:

\begin{itemize}
	\item They should have experience with performing
	\item They should have a basic understanding of how voice effects works
\end{itemize}

The target group consists of performers that know about voice effects. They should not play an instrument or similar while performing, because they must be able to use their body for controlling the effects. There is no specific genre or type of performer as the only criteria is that they know the technicalities behind performing.

People who fulfil these criteria could be singers, stage actors and stand-up comedians.