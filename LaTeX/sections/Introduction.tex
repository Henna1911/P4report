\chapter{Introduction}\label{ch:Intro}

\section{Initial Problem Statement}

\subsection{Motivation}

It is possible to use voice effects while performing. Many useful effects for performing exist, and it is possible to change the parameters of these effects to one's liking in real time. 
A problem can be applying an effect and/or changing the effect parameters while performing without having a big control board in front of the performer which may cause a disruption of the performance.

\subsection{Statement}

\textit{How does one create a system that applies voice effects to a voice in real-time?}


\subsection{Research Questions}\label{sub:ResearchQ}

\begin{itemize}
	\item What are the most common voice effects?
	\begin{itemize}
		\item What are the limits of the effects? \todo{is rephrasing needed here? Olivier thought something should be done}
	\end{itemize}
	\item Does any existing technology use body gestures or sensors to apply effects?
\end{itemize}

\subsection{Target Group}
The criteria for the target group in this project are:
\begin{itemize}
	\item Should be able to sing 
	\item Should know the basic theory of singing
	\item Preferably should know about voice effects
\end{itemize}

The target group consists of singers that know about voice effects. They should not play an instrument while singing because they must be able to use their body for controlling the effects. There is no specific genre or type of singer as the only criteria is that they know the technicalities behind singing.

People who fulfil these criteria could for example be solo singers, choirs and band singers.