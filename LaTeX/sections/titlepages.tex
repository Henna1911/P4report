\pdfbookmark[0]{English title page}{label:titlepage_en}
\aautitlepage{%
  \englishprojectinfo{
    Hand Recognition Game Control using Image Processing %title
  }{%
    Visual Computing - Human Perception %theme
  }{%
    Autumn Semester 2015 %project period
  }{%
    MTA 15335 % project group
  }{%
    %list of group members
    Alex Bo Mikkelsen\\ 
    Allan Schjørring\\
    Daniel Agerholm Johansen\\
	Frederik Homann Østergaard\\    
    Jens Viggo Jensen\\
    Kaspar Pawlak\\
    Kasper Lind Vildner Pedersen
    
  }{%
    %list of supervisors
    Mohammad Ahsanul Haque
  }{%
    8 % number of printed copies
  }{%
    December 17, 2015 % date of completion
  }%
}{%department and address
  \textbf{Media Technology}\\
  Rendsburggade 14\\
  DK-9000 Aalborg
}{% the abstract
  This paper examines if hand signs can be used to control a game with the help of a webcam. In the research chapter, image processing methods are found, different technologies on capturing data are explained, and game genres are explored. In the final problem statement, it is chosen to work on a 2D platform game, that should be controlled with three or more hand signs. The game concept and image processing theories are then explained, followed by a development chapter. The image processing program is made with C++ and the OpenCV library, whereas the game itself is made with Unity 5. The program can recognize four handsigns, and the game has four levels. In the internal test, it was found that the system has a small amount of delay. The user tests showed hand signs can be used to control the game, but some of the levels were difficult to complete, and that the lighting environment has a big impact on the hand sign recognition. 

}

\cleardoublepage
