\pdfbookmark[0]{English title page}{label:titlepage_en}
\aautitlepage{%
  \englishprojectinfo{
    Real-Time Voice Manipulation Using a Wearable Device %title
  }{%
    Sound Computing and Sensor Technology %theme
  }{%
    Spring Semester 2016 %project period
  }{%
    MTA 16440 % project group
  }{%
    %list of group members
    Alex Bo Mikkelsen\\ 
    Allan Schjørring\\
    Daniel Agerholm Johansen\\
    Didzis Gailitis\\
	Liv Arleth\\    
    Sebastian Laczek Nielsen
    
  }{%
    %list of supervisors
    Olivier Lartillot
  }{%
    8 % number of printed copies
  }{%
    May 26, 2016 % date of completion
  }%
}{%department and address
  \textbf{Media Technology}\\
  Rendsburggade 14\\
  DK-9000 Aalborg
}{% the abstract
  Today the use of voice effects are common in music. It can however be difficult to change these in real time during a performance. This problem is addressed in this project by focusing on creating a more intuitive and simple  solution than the ones that already exist.
In this project a wearable device, to put on the hand, is created to apply voice effects to a performance in real time. Different effects has been researched and pitch shift and harmonisation were eventually chosen to be implemented. This was done by using a gyroscope to track hand movement which is connected to an Arduino. The audio is processed in Pure Data. It was concluded through testing that the system was fully functional. It is not possible to draw a conclusion from the user test, since the participants were not optimal.

}

\cleardoublepage
